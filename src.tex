\documentclass[12pt]{beamer}

\usepackage[T1]{fontenc}
\usepackage[francais]{babel}

\usepackage{lmodern}
\usepackage{fontspec}
\usepackage{graphicx}

\usepackage{hyperref}
\usepackage{minted}
\usepackage{tikz}

% BEGIN STYLE

\setbeameroption{hide notes}
\setbeamertemplate{note page}[plain]

\usetheme{default}
\beamertemplatenavigationsymbolsempty
\hypersetup{pdfpagemode=UseNone}

\setmonofont{Inconsolata}

\usefonttheme{professionalfonts}
\usefonttheme{serif}
\usepackage{fontspec}
\setmainfont{Helvetica Neue}
\setbeamerfont{note page}{family*=pplx,size=\footnotesize}

\definecolor{latexblue}{RGB}{87,102,181}

\definecolor{foreground}{RGB}{255,255,255}
\definecolor{background}{RGB}{51,86,65}
\definecolor{title}{RGB}{255,255,255}
\definecolor{gray}{RGB}{159,255,197}
\definecolor{subtitle}{RGB}{159,255,197}
\definecolor{hilight}{RGB}{102,255,204}
\definecolor{vhilight}{RGB}{255,111,207}

\useinnertheme{rectangles}

\setbeamercolor{background canvas}{bg=background}
\setbeamercolor{titlelike}{fg=title}
\setbeamercolor{subtitle}{fg=subtitle}
\setbeamercolor{institute}{fg=gray}
\setbeamercolor{normal text}{fg=foreground,bg=background}
\setbeamercolor{item projected}{bg=foreground, fg=background}
\setbeamercolor{structure}{fg=foreground}
\setbeamercolor{subitem}{fg=foreground}
\setbeamercolor{itemize/enumerate subbody}{fg=foreground}

\setbeamertemplate{itemize subitem}{{\textendash}}

\setbeamerfont{itemize/enumerate subbody}{size=\footnotesize}
\setbeamerfont{itemize/enumerate subitem}{size=\footnotesize}

\setbeamertemplate{footline}{
    \raisebox{5pt}{\makebox[\paperwidth]{\hfill\makebox[20pt]{\color{gray}
\scriptsize\insertframenumber}}}\hspace*{5pt}}

\addtobeamertemplate{note page}{\setlength{\parskip}{12pt}}

\newminted{csharp}{bgcolor=gray!25,fontsize=\footnotesize,mathescape}

% END STYLE

\title[Introduction au C\#]{\textbf{Introduction au C\#}}

\author{
    Valentin 'toogy' Iovene \\
    Quentin 'Neodyblue' Coelho
}

\institute{\includegraphics[scale=0.38]{img/logos.png}}

\date{}

\begin{document}

{
    \setbeamertemplate{footline}{} % no page number here
    \frame{
        \titlepage
    }
}

\section*{Introduction}

\begin{frame}
    \frametitle{La conférence}
    \tableofcontents[pausesections]
\end{frame}

\section{Organiser son projet et maintenir son code}

\begin{frame}
    \begin{center}
        \vspace{1cm}
        {\Large \textbf{1.} Organiser son projet\\
        et maintenir son code \\}
    \end{center}
\end{frame}

\begin{frame}
    \begin{center}
        \vspace{1cm}
        {\large Organiser son \textbf{projet}}
    \end{center}
\end{frame}

\begin{frame}[fragile]
    \begin{center}
        \vspace{1cm}
        {\large Réfléchir avant d'agir (de coder)} \\
    \end{center}
\end{frame}

\begin{frame}
    \vspace{1cm}
    {\large \textbf{Avoir les idées claires, savoir où on va}} \\
    \begin{itemize}
        \pause \item Brainstorming : trouver ce qu'on \textbf{veut} faire \\
        \pause \item Fixer un but \emph{concret} à atteindre (le produit)
        \pause \item Représenter son projet visuellement
    \end{itemize}
\end{frame}

\begin{frame}
    \vspace{1cm}
    {\large \textbf{Découper son projet en fonctionnalités}} \\
    \begin{itemize}
        \pause\item Menu \\
        \pause\item Minimap \\
        \pause\item Map \\
        \pause\item Unités \\
        \begin{itemize}
            \pause\item Ouvrier \\
            \begin{itemize}
                \pause\item Récolte \\
                \pause\item Construction de bâtiments \\
            \end{itemize}
            \pause\item Combats \\
            \pause\item Déplacements \\
        \end{itemize}
        \pause\item Système de ressources \\
        \pause\item Bâtiments \\
        \begin{itemize}
            \pause\item File d'attente de création d'unités \\
            \pause\item Point de ralliement \\
        \end{itemize}
    \end{itemize}
\end{frame}

\begin{frame}
    \vspace{1cm}
    {\large \textbf{Priorisation (valeur ajoutée + dépendance)}} \\
    \begin{itemize}
        \item Map \\
        \item Unités \\
        \item Unités:Déplacements \\
        \item Système de ressources \\
        \item Bâtiments \\
        \item Unités:Ouvrier:Récolte \\
        \item Unités:Ouvrier:Construction de bâtiments \\
        \item Unités:Combats \\
        \item Bâtiments:File d'attente de création d'unités \\
        \item Bâtiments:Point de ralliement \\
        \item Minimap \\
        \item Menu \\
    \end{itemize}
\end{frame}

%{ % all template changes are local to this group.
    %\setbeamertemplate{background canvas}{\includegraphics[height=\paperheight]{img/baby.jpg}}
    %\begin{frame}
        %\begin{center}
            %{\Huge \textbf{KNIFE THE BABY}}
        %\end{center}
    %\end{frame}
%}

\begingroup
\setbeamercolor{background canvas}{bg=title}
\begin{frame}
    \begin{center}
        \vspace{1cm}
        {\Large\color{background} Maintenir son \textbf{\{code\}}}
    \end{center}
\end{frame}
\endgroup

\end{document}
