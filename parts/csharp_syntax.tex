\section{Syntaxe C\#}

\begingroup
\setbeamercolor{background canvas}{bg=title}
\begin{frame}
    \begin{center}
        \vspace{1cm}
        {\Huge\color{background}
            Syntaxe C\#
        }
    \end{center}
\end{frame}
\endgroup

\begin{frame}
  \begin{center}
    \vspace{1cm}
    {\Huge Les variables}
  \end{center}
\end{frame}

\setbeamercovered{transparent}
\begin{frame}[c]
  \frametitle{Déclaration}

  \begin{center}
    \begin{itemize}
      \item<+-> <type> <name>;
      \item<+-> <type> <name> = <valeur>;
      \item<+-> <type>[ ] <name>;
      \item<+-> <type>[ ] <name> = new <type>[<size>];
    \end{itemize}
  \end{center}
\end{frame}

\begin{frame}[fragile]
  \frametitle{Déclaration}

  \begin{center}{\large Exemples}\end{center}
  \begin{csharpcode*}{fontsize=\normalsize}
    int i;
    string str;
    int a = 5;
    long verylong = 100000000000000;
    float val = 6.4f;
    double d = 4.2;
    string str2 = "Hello";
    int[] datas = new int[10];
  \end{csharpcode*}
\end{frame}

\begin{frame}[fragile]
  \frametitle{Manipulation}

  \begin{center}{\large Nombres}\end{center}
  \begin{csharpcode*}{fontsize=\scriptsize}
    int[] results = new int[2];
    int a = 4;
    bool vrai = true;
    bool faux = false;
    int b = 6;
    a = -1;
    results[0] = a + b; // [0] = 5
    b = 3;
    results[1] = a * b; // [1] = -3
  \end{csharpcode*}

  \pause

  \begin{center}{\large Strings}\end{center}
  \begin{csharpcode*}{fontsize=\scriptsize}
    string name = "Neodyblue";
    string text = "Welcome " + name + " !";// "Welcome Neodyblue !"
  \end{csharpcode*}
\end{frame}

\begin{frame}
  \begin{center}
    \vspace{1cm}
    {\Large Les structures de contrôle}
  \end{center}
\end{frame}

\begin{frame}[fragile]
  \frametitle{Les opérateurs de comparaison}

  \begin{itemize}
    \item<+-> == : 3 == 3
    \item<+-> != : 1 != 2
    \item<+-> < : 1 < 3
    \item<+-> > : 3 > 1
    \item<+-> <= : 1 <= 1
    \item<+-> >= : 4 >= 2
  \end{itemize}
\end{frame}

\begin{frame}[fragile]
  \frametitle{Le Test.}

  \begin{csharpcode*}{fontsize=\normalsize}
    Console.WriteLine("Say my name !");
    string name = Console.ReadLine();

    if (name == "Heisenberg")
    {
      Console.WriteLine("You're goddamn right.");
    }
    else
    {
      Console.WriteLine("You lose !");
    }
  \end{csharpcode*}
\end{frame}

\begin{frame}
  \begin{center}
    \vspace{1cm}
    {\Large Les boucles}
  \end{center}
\end{frame}

\begin{frame}[fragile]

  \begin{csharpcode*}{fontsize=\normalsize}
    for (<init>; <booleen>; <step>)
    {
      // Do something
    }
  \end{csharpcode*}
\end{frame}

\begin{frame}[fragile]

  \begin{csharpcode*}{fontsize=\normalsize}
    for (int i = 0; i < 10; ++i)
    {
      // This code will be executed 10 times
      // i values : 0, 1, 2, ..., 8, 9
    }
  \end{csharpcode*}
\end{frame}

\begin{frame}[fragile]

  \begin{csharpcode*}{fontsize=\normalsize}
    while (<booleen>)
    {
      // Do something
    }
  \end{csharpcode*}
\end{frame}

\begin{frame}[fragile]

  \begin{csharpcode*}{fontsize=\normalsize}
    string str = "";
    while (str.Length != 10)
    {
      str += "A";
    }
  \end{csharpcode*}
\end{frame}

\begin{frame}[fragile]
  \begin{csharpcode*}{fontsize=\normalsize}
    do
    {
      // Do something at least 1 time
    } while (<booleen>);
  \end{csharpcode*}
\end{frame}

\begin{frame}[fragile]
  \begin{csharpcode*}{fontsize=\normalsize}
    string str = "";
    do
    {
      str = Console.ReadLine();
      // Do something with str
    } while (str != "exit");
  \end{csharpcode*}
\end{frame}

\begin{frame}
  \begin{center}
    \vspace{1cm}
    {\Large Factorisation : les fonctions}
  \end{center}
\end{frame}

\begin{frame}[fragile]
  \frametitle{Déclaration de fonction}

  \begin{csharpcode*}{fontsize=\normalsize}
    <type> <name>(<parameter>)
    {
      // Code
    }
  \end{csharpcode*}
\end{frame}

\begin{frame}[fragile]
  \begin{columns}[c]
    \column{2.3in}
    \begin{csharpcode*}{fontsize=\normalsize}
      int sum(int a, int b)
      {
        return a + b;
      }
    \end{csharpcode*}

    \pause

    \begin{csharpcode*}{fontsize=\normalsize}
      bool is_even(int n)
      {
        return n % 2 == 0;
      }
    \end{csharpcode*}

    \pause

    \column{2.3in}
    \begin{csharpcode*}{fontsize=\normalsize}
      int div(int c, int d)
      {
        if (d == 0)
        {
          // FAIL
        }
        else
          return c / d;
      }
    \end{csharpcode*}
  \end{columns}
\end{frame}

\begin{frame}[fragile]
  \begin{csharpcode*}{fontsize=\normalsize}
    int x = 5;
    int y = 11;

    int result_sum = sum(x, y); // result_sum = 16
    int result_div = div(x, y); // result_div = 0
    bool even = is_even(y); // even = false
  \end{csharpcode*}
\end{frame}

\begin{frame}[fragile]
  \frametitle{Surcharge}

  \begin{csharpcode*}{fontsize=\normalsize}
    void print(int n)
    {
      Console.WriteLine (n);
    }

    void print(int n, string msg)
    {
      Console.WriteLine (msg + " " + n);
    }
  \end{csharpcode*}
\end{frame}

\begin{frame}
  \begin{center}
    \vspace{1cm}
    {\Huge Les collections}
  \end{center}
\end{frame}

\begin{frame}[fragile]
  \frametitle{Listes}

  \begin{csharpcode*}{fontsize=\normalsize}
    List<<type>> <name> = new List<<type>>();
  \end{csharpcode*}
\end{frame}

\begin{frame}[fragile]
  \begin{csharpcode*}{fontsize=\normalsize}
    List<string> words = new List<string>();
  \end{csharpcode*}

  \pause

  \begin{center}{\large Ajout et suppression}\end{center}
  \begin{csharpcode*}{fontsize=\normalsize}
    words.Add("hello");
    words.Add("world");
    words.Add("test");

    words.Remove("test");
  \end{csharpcode*}
\end{frame}

\begin{frame}[fragile]
  \frametitle{Parcourir une collection}

  \begin{csharpcode*}{fontsize=\normalsize}
    foreach (<type> <name> in <collection>)
    {
      // Do something with <name>
    }

    for (int i = 0; i < <collection>.Count; ++i)
    {
      // Do something with <name> and its index
    }
  \end{csharpcode*}
\end{frame}

\begin{frame}[fragile]
  \frametitle{Parcourir une collection}

  \begin{csharpcode*}{fontsize=\normalsize}
    foreach (string word in words)
    {
      Console.WriteLine(word);
    }

    for (int i = 0; i < words.Count; ++i)
    {
      Console.WriteLine("[" + i + "]: " + words[i]);
    }
  \end{csharpcode*}
\end{frame}


