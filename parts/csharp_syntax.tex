\section{Syntaxe C\#}

\begingroup
\setbeamercolor{background canvas}{bg=title}
\begin{frame}
    \begin{center}
        \vspace{1cm}
        {\Large\color{background}
            Syntaxe C\#
        }
    \end{center}
\end{frame}
\endgroup

\begin{frame}
  \begin{center}
    \vspace{1cm}
    Les variables
  \end{center}
\end{frame}

\setbeamercovered{transparent}
\begin{frame}[c]
  \frametitle{Déclaration}

  \begin{center}
    \begin{itemize}
      \item<+-> <type> <name>;
      \item<+-> <type> <name> = <valeur>;
      \item<+-> <type>[] <name>;
      \item<+-> <type>[] <name> = new <type>[<size>];
    \end{itemize}
  \end{center}
\end{frame}

\begin{frame}[fragile]
  \frametitle{Déclaration}

  \begin{center}{\large Exemples}\end{center}
    \begin{csharpcode*}{fontsize=\scriptsize}
      int i;
      string str;
      int a = 5;
      long verylong = 100000000000000;
      float val = 6.4f;
      double d = 4.2;
      string str2 = "Hello";
      int[] datas = new int[10];
    \end{csharpcode*}
  \end{frame}

\begin{frame}[fragile]
  \frametitle{Manipulation}

  \begin{center}{\large Nombres}\end{center}
  \begin{csharpcode*}{fontsize=\scriptsize}
    int[] results = new int[2];
    int a = 4;
    bool vrai = true;
    bool faux = false;
    int b = 6;
    a = -1;
    results[0] = a + b; // [0] = 5
    b = 3;
    results[1] = a * b; // [1] = -3
  \end{csharpcode*}

  \pause

  \begin{center}{\large Strings}\end{center}
  \begin{csharpcode*}{fontsize=\scriptsize}
    string name = "Neodyblue";
    string text = "Welcome " + name + " !"; // "Welcome Neodyblue !"
  \end{csharpcode*}
\end{frame}

\begin{frame}[fragile]
  \frametitle{Structures de contrôle}

  \begin{center}{\large Opérateurs de comparaison}\end{center}
  \begin{itemize}
    \item<+-> == : 3 == 3
    \item<+-> != : 1 != 2
    \item<+-> < : 1 < 3
    \item<+-> > : 3 > 1
    \item<+-> <= : 1 <= 1
    \item<+-> >= : 4 >= 2
  \end{itemize}
\end{frame}

\begin{frame}[fragile]
  \frametitle{Structures de contrôle}

  \begin{center}{\large Le test}\end{center}
  \begin{csharpcode*}{fontsize=\scriptsize}
    if (<booléen>)
    {
      // true
    }
    else
    {
      // false
    }
  \end{csharpcode*}
\end{frame}

\begin{frame}[fragile]
  \frametitle{Structures de contrôle}

  \begin{csharpcode*}{fontsize=\scriptsize}
    Console.WriteLine("Say my name !");
    string name = Console.ReadLine();

    if (name == "Heisenberg")
    {
      Console.WriteLine("You're goddamn right.");
    }
    else
    {
      Console.WriteLine("You lose !");
    }
  \end{csharpcode*}
\end{frame}

\begin{frame}[fragile]
  \frametitle{Les boucles}

  \begin{center}{\large Le for}\end{center}
  \begin{csharpcode*}{fontsize=\scriptsize}
    for (<init>; <booléen>; <step>)
    {
      // Do something
    }
  \end{csharpcode*}
\end{frame}

\begin{frame}[fragile]
  \frametitle{Les boucles}

  \begin{csharpcode*}{fontsize=\scriptsize}
    for (int i = 0; i < 10; ++i)
    {
      // This code will be executed 10 times
      // i values : 0, 1, 2, ..., 8, 9
    }
  \end{csharpcode*}
\end{frame}

\begin{frame}[fragile]
  \frametitle{Les boucles}

  \begin{center}{\large Le while}\end{center}
  \begin{csharpcode*}{fontsize=\scriptsize}
    while (<booléen>)
    {
      // Do something
    }
  \end{csharpcode*}
\end{frame}

\begin{frame}[fragile]
  \frametitle{Les boucles}

  \begin{csharpcode*}{fontsize=\scriptsize}
    string str = "";
    while (str.Length != 10)
    {
      str += "A";
    }
  \end{csharpcode*}
\end{frame}

\begin{frame}[fragile]
  \frametitle{Les boucles}

  \begin{center}{\large Le do while}\end{center}
  \begin{csharpcode*}{fontsize=\scriptsize}
    do
    {
      // Do something at least 1 time
    } while (<booléen>);
  \end{csharpcode*}
\end{frame}

\begin{frame}[fragile]
  \frametitle{Les boucles}

  \begin{csharpcode*}{fontsize=\scriptsize}
    string str = "";
    do
    {
      str = Console.ReadLine();
      // Do something with str
    } while (str != "exit");
  \end{csharpcode*}
\end{frame}
